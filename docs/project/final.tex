\documentclass{article}

\usepackage[T1]{fontenc}
\usepackage[utf8]{inputenc}
\usepackage[polish]{babel}
\usepackage{enumerate}
\usepackage{amsmath}
\usepackage{tikz}
\usepackage[a4paper]{geometry}

\title{PSZT - Sterowanie ptakiem}
\author{
  Marcin Puc\\
  \and
  Piotr Komorowski\\
  \and
  Ireneusz Wróbel
}
\date{}

\setlength{\parskip}{1em}

\begin{document}

\maketitle

\section {Opis projektu}

Projekt \emph{Sterowanie ptakiem} obejmuje stworzenie klona gry \emph{Flappy
  Bird} oraz sterownika do tej gry opartego na logice rozmytej.

Gra polega na sterowaniu ptakiem w taki sposób, żeby omijać przeszkody -
pojawiające się na horyzoncie rury, między którymi znajdują się prześwity, przez
które ptak może przelecieć. Postać posiada utrudniony model latania, w którym
może tylko biernie spadać w dół lub wykonywać pojedyncze podskoki w powietrzu.

\section {Decyzje projektowe}

\subsection{Zmienne}

W celu optymalnego sterowania został zaprojektowany silnik rozmyty przyjmujący
dwie zmienne wejściowe:

\begin{enumerate}
\item \emph{XPos} - odległość lewej krawędzi ptaka od prawej krawędzi najbliższej rury
  \begin{enumerate}
  \item \textbf{Unsafe}

    ptak musi podskakiwać żeby wyrównywać poziom - znajduje się
    na lewo od rury lub nad nią

  \item \textbf{Safe}

    ptak znajduje się na tyle blisko ($x_{safe}$) krańca rury, że nie musi już podskakiwać -
    w naturalny sposób przeleci tuż nad krawędzią rury podczas spadania 

  \end{enumerate}
\item \emph{Altitude} - odległość dolnej krawędzi ptaka od ustalonego poziomu
  bezpiecznego nad górną krawędzią najbliższej rury
  \begin{enumerate}
  \item \textbf{Low}

    ptak znajduje się poniżej poziomu bezpiecznego - konieczny jest podskok

  \item \textbf{Ok}

    ptak znajduje się powyżej lub na poziomie bezpiecznym - może swobodnie spadać

  \end{enumerate}
\end{enumerate}

Zmienna wyjściowa \emph{Movement} reprezentuje wartość sygnału sterującego i posiada
następujące terminy lingwistyczne:

\begin{enumerate}[a)]
\item \textbf{Flap}

  podskok w powietrzu

\item \textbf{Fall}

  swobodne spadanie

\end{enumerate}

Przyjmując, że ${alt}_{min}$ - ustalony pułap bezpieczny nad górną krawędzią
rury, $v_{hor}$ - prędkość lotu ptaka w poziomie, $v_{fall}$ - prędkość
spadania, możemy obliczyć $x_{safe}$:

\begin{equation}
  \nonumber
  x_{safe} = {alt}_{min} \cdot \frac{v_{hor}}{v_{fall}}
\end{equation}

\begin{center}

  \begin{tikzpicture}
    \filldraw [fill=green, draw=black] (-1,0) rectangle (1,2);
    \filldraw [fill=yellow, draw=black] (0,4) rectangle (1.5, 5);
    \draw [dashed] (-1.25,3) -- (1.25,3);
    \draw [<->] (-1,2) -- (-1,3) node[left, midway]{${alt}_{min}$};
    \draw [<->] (-0.1,3) -- (-0.1,4) node[left, midway]{Altitude};
    \draw [orange, dashed] (0,2) -- (0,4);
    \draw [purple, dashed] (1,2) -- (1,4);
    \draw [<->] (0,2.75) -- (1,2.75) node[below, midway]{XPos};
    \filldraw (0.75, 4.5) circle (2pt);
    \draw [->, thick] (0.75,4.5) -- (1.75,4.5) node[right]{$v_{hor}$};
    \draw [->, thick] (0.75,4.5) -- (0.75,3.75) node[below]{$v_{fall}$};
  \end{tikzpicture}

\end{center}

\begin{figure}

  \centering
  \begin{tikzpicture}
    \draw [->, thick] (0,0) -- (4,0) node[right] {$x$};
    \draw [->, thick] (0,0) -- (0,2) node[above] {$\mu(x)$};
    \draw (0,1) node[left]{1};
    \draw [purple] (0,1) -- (0.99,1) node[above, midway]{Safe} -- (0.99,0);
    \draw (1,0) node[below]{$x_{safe}$};
    \draw [orange] (1,0) -- (1,1) -- (4,1) node[above, midway]{Unsafe};
  \end{tikzpicture}
  \caption{Przyporządkowanie dla zmiennej \emph{XPos}}

  \centering
  \begin{tikzpicture}
    \draw [->, thick] (-2,0) -- (2,0) node[right] {$x$};
    \draw [->, thick] (0,0) -- (0,2) node[above] {$\mu(x)$};
    \draw [red] (-2,1) -- (0,1) node[above, midway]{Low};
    \draw (0,1) node[left]{1};
    \draw [green] (0,1) -- (2,1) node[above, midway]{Ok};
  \end{tikzpicture}
  \caption{Przyporządkowanie dla zmiennej \emph{Altitude}}

  \centering
  \begin{tikzpicture}
    \draw [->, thick] (-2,0) -- (2,0) node[right] {$x$};
    \draw [->, thick] (0,0) -- (0,2) node[above] {$\mu(x)$};
    \draw (0,1) node[left]{1};
    \draw [blue] (-1,0) -- (-1,1) -- (0,1) node[above, midway]{Fall};
    \draw (-1,0) node[below]{-1};
    \draw [brown] (0,1) -- (1,1) node[above, midway]{Flap} -- (1,0);
    \draw (1,0) node[below]{1};
  \end{tikzpicture}
  \caption{Przyporządkowanie dla zmiennej \emph{Movement}}

\end{figure}

\section {Opis użytkowania}

Dostarczony kod źródłowy zawiera plik \texttt{FuzzyBird.pro}. Aby skompilować
program, należy otworzyć ten plik za pomocą środowiska QtCreator, a następnie
skompilować i uruchomić w nim projekt.

Program po uruchomieniu wyświetla menu główne, w którym możemy wybrać grę w
jednym z dwóch trybów: \emph{Play} (sterowanie klawiaturą) lub \emph{AI
  Controlled} (rozgrywka kontrolowana przez sterownik). W przypadku trybu
klawiatury klawisz Spacja służy do wykonywania podskoków w powietrzu. Wciśniecie
klawisza Escape podczas gry powoduje powrót do głównego menu.

W przypadku przegrania rozgrywki zostaje wyswietlony ekran końca gry, na którym widzimy swój
ostateczny wynik i możemy wybrać opcję zagrania ponownie w tym samym trybie, bądź powrotu
do menu głównego.

\section {Struktura programu}

\subsection {Moduły}

Program składa się z następujących modułów:

\begin{enumerate}
\item \textbf{EngineProperties}

  struktury służące do opisu silnika: specyfikacja zmiennych, parametrów itp.

\item \textbf{EngineFactory}

  fabryka silników, przyjmuje obiekt \emph{EngineProperties} i tworzy silnik o
  opisanych przez niego właściwościach

\item \textbf{Engine}

  główny moduł wykonawczy, zbudowany z poniższych komponentów, zarządza potokiem
  przetwarzania wartości skalarnych

\item \textbf{InputVar}

  obiekt reprezentujący zmienną wejściową, posiada zbiór terminów
  lingwistycznych

\item \textbf{OutputVar}

  obiekt reprezentujący zmienną wyjściową, posiada zbiór terminów
  lingwistycznych oraz dziedzinę wartości skalarnych

\item \textbf{Term}

  obiekt abstrakcyjny reprezentujący termin lingwistyczny oraz jego
  funkcję przyporządkowania

\item \textbf{Norm}

  obiekt abstrakcyjny reprezentujący operację wykonywaną na wartościach
  skalarnych i terminach lingwistycznych

\item \textbf{AndExpr}

  obiekt reprezentujący warunek reguły (złączenie terminów operatorem koniunkcji)

\item \textbf{Implicator}

  obiekt reprezentujący pojedynczą regułę: warunek i konkluzję, wykonuje
  operację implikacji logicznej (iloczyn wartości przyporządkowania warunku i
  terminu konkluzji)

\item \textbf{Aggregator}

  obiekt reprezentujący agreagację wielu reguł w jedną wartość przyporządkowania
  za pomocą operatora sumy logicznej wartości implikatorów

\item \textbf{RuleBlock}

  obiekt zarządzający regułami: wewnętrznie składa się z obiektów Aggregator,
  Implicator i AndExpr

\item \textbf{Defuzzifier}

  obiekt dokonujący defuzyfikacji, czyli wyszukiwania reprezentanta (wyniku) zmiennej
  wyjściowej poprzez wyliczanie środka ciężkości funkcji zagregowanej metodą całkowania

\end{enumerate}

\subsection {Funkcje przyporządkowania}

\begin{enumerate}
\item \textbf{Binary}

  binarne przyporządkowanie, po jednej stronie zbocza przyjmuje wartość
  niezerową, a po drugiej - zero

\item \textbf{Ramp}

  działa podobnie jak \emph{Binary}; zbocze funkcji jest pochyłe, a nie pionowe

\item \textbf{Rectangle}

  przyjmuje wartość niezerową dla określonego przedziału argumentów i zero dla
  pozostałych wartości

\item \textbf{Trapezoid}

  działa podobnie jak \emph{Rectangle}; oba zbocza są pochyłe, a nie pionowe

\item \textbf{Triangle}

  przyporządkowanie o kształcie trójkątnym, wartości przyjmowane dążą liniowo od zera do wierzchołka
  trójkąta z lewej i z prawej strony oraz wynoszą zero poza jego obszarem

\end{enumerate}

\subsection {Operatory Norm}

\begin{enumerate}
\item \emph{TNorm}

  operatory reprezentujące abstrakcyjną operację iloczynu:

  \begin{enumerate}
  \item \textbf{AlgebraicProduct}

    zwraca iloczyn algebraiczny argumentów

  \item \textbf{Minimum}

    zwraca najmniejszą wartość spośród argumentów

  \end{enumerate}

\item \emph{SNorm}

  operatory reprezentujące abstrakcyjną operację sumy:

  \begin{enumerate}
  \item \textbf{BoundedSum}

    zwraca sumę algebraiczną argumentów ograniczoną z góry przez 1

  \item \textbf{Maximum}

    zwraca najwyższą wartość spośród argumentów

  \end{enumerate}

\end{enumerate}

\section {Wnioski}

Z obserwacji przeprowadzonych na różnych poziomach trudności gry (parametry
szybkości, odległości między obiektami itp.) wynika, że zaprojketowany powyżej
silnik osiąga zadowalające rezultaty. Sterownik przegrywa tylko w sytuacjach, w
których przegrana jest nieunikniona, natomiast w pozostałych przypadkach
przeszkody są płynnie omijane.

\end{document}